\begin{abstract}
\noindent
The site frequency spectrum (SFS) is an important summary statistic in population genetics used for inference on demographic history and selection.
However, estimation of the SFS from called genotypes introduce bias when working with low-coverage sequencing data.
Methods exist for addressing this issue, but sometimes suffer from two problems.
First, they can have very high computational demands, to the point that it may not be possible to run estimation for genome-scale data.
Second, existing methods are prone to overfitting, especially for multi-dimensional SFS estimation.
In this article, we present a stochastic expectation-maximisation algorithm for inferring the SFS from NGS data that addresses these challenges.
We show that this algorithm greatly reduces runtime and enables estimation with constant, trivial RAM usage.
Further, the algorithm reduces overfitting and thereby improves downstream inference.
An implementation is available at \shorturl{github.com/malthesr/winsfs}.
\end{abstract}
